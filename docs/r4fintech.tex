\documentclass[]{book}
\usepackage{lmodern}
\usepackage{amssymb,amsmath}
\usepackage{ifxetex,ifluatex}
\usepackage{fixltx2e} % provides \textsubscript
\ifnum 0\ifxetex 1\fi\ifluatex 1\fi=0 % if pdftex
  \usepackage[T1]{fontenc}
  \usepackage[utf8]{inputenc}
\else % if luatex or xelatex
  \ifxetex
    \usepackage{mathspec}
  \else
    \usepackage{fontspec}
  \fi
  \defaultfontfeatures{Ligatures=TeX,Scale=MatchLowercase}
\fi
% use upquote if available, for straight quotes in verbatim environments
\IfFileExists{upquote.sty}{\usepackage{upquote}}{}
% use microtype if available
\IfFileExists{microtype.sty}{%
\usepackage{microtype}
\UseMicrotypeSet[protrusion]{basicmath} % disable protrusion for tt fonts
}{}
\usepackage[margin=1in]{geometry}
\usepackage{hyperref}
\hypersetup{unicode=true,
            pdftitle={R for FinTech},
            pdfauthor={Jasmine Dumas},
            pdfborder={0 0 0},
            breaklinks=true}
\urlstyle{same}  % don't use monospace font for urls
\usepackage{natbib}
\bibliographystyle{apalike}
\usepackage{longtable,booktabs}
\usepackage{graphicx,grffile}
\makeatletter
\def\maxwidth{\ifdim\Gin@nat@width>\linewidth\linewidth\else\Gin@nat@width\fi}
\def\maxheight{\ifdim\Gin@nat@height>\textheight\textheight\else\Gin@nat@height\fi}
\makeatother
% Scale images if necessary, so that they will not overflow the page
% margins by default, and it is still possible to overwrite the defaults
% using explicit options in \includegraphics[width, height, ...]{}
\setkeys{Gin}{width=\maxwidth,height=\maxheight,keepaspectratio}
\IfFileExists{parskip.sty}{%
\usepackage{parskip}
}{% else
\setlength{\parindent}{0pt}
\setlength{\parskip}{6pt plus 2pt minus 1pt}
}
\setlength{\emergencystretch}{3em}  % prevent overfull lines
\providecommand{\tightlist}{%
  \setlength{\itemsep}{0pt}\setlength{\parskip}{0pt}}
\setcounter{secnumdepth}{5}
% Redefines (sub)paragraphs to behave more like sections
\ifx\paragraph\undefined\else
\let\oldparagraph\paragraph
\renewcommand{\paragraph}[1]{\oldparagraph{#1}\mbox{}}
\fi
\ifx\subparagraph\undefined\else
\let\oldsubparagraph\subparagraph
\renewcommand{\subparagraph}[1]{\oldsubparagraph{#1}\mbox{}}
\fi

%%% Use protect on footnotes to avoid problems with footnotes in titles
\let\rmarkdownfootnote\footnote%
\def\footnote{\protect\rmarkdownfootnote}

%%% Change title format to be more compact
\usepackage{titling}

% Create subtitle command for use in maketitle
\newcommand{\subtitle}[1]{
  \posttitle{
    \begin{center}\large#1\end{center}
    }
}

\setlength{\droptitle}{-2em}
  \title{R for FinTech}
  \pretitle{\vspace{\droptitle}\centering\huge}
  \posttitle{\par}
  \author{Jasmine Dumas}
  \preauthor{\centering\large\emph}
  \postauthor{\par}
  \predate{\centering\large\emph}
  \postdate{\par}
  \date{2016-10-09}

\usepackage{booktabs}

\begin{document}
\maketitle

{
\setcounter{tocdepth}{1}
\tableofcontents
}
\begin{figure}[htbp]
\centering
\includegraphics{cover.png}
\caption{}
\end{figure}

\chapter{Welcome}\label{welcome}

Welcome to the guidebook \textbf{R for FinTech}! This guidebook will
include examples, tutorials, and highlight essential R packages for Data
Science. The organization of this guide is inspired by the book
\href{http://r4ds.had.co.nz/}{\textbf{R for Data Science}} from Garrett
Grolemund and Hadley Wickham which explores each step of the data
science process from acquiring data to communicating the outputs.

This guidebook has emphasis on
\href{https://en.wikipedia.org/wiki/Financial_technology}{``FinTech''}
or Financial Technology applications in data analysis. When starting out
in a new industry or a new programming langauge like R, it can be
difficult to learn about how to apply industry-specific methods given
the vast amount of R packages available and the sparcity of examples

\section{Prerequisites}\label{prerequisites}

\textbf{If you don't already have R or RStudio:}

\begin{itemize}
\tightlist
\item
  Download R at \url{https://www.r-project.org/alt-home/}
\item
  Download RStudio at
  \url{https://www.rstudio.com/products/rstudio/download/}
\end{itemize}

\chapter{Import}\label{import}

\chapter{Tidy}\label{tidy}

\chapter{Transform}\label{transform}

\chapter{Visualization}\label{viz}

\chapter{Model}\label{model}

\chapter{Communicate}\label{communicate}

\chapter{References}\label{references}

\bibliography{packages.bib,book.bib}


\end{document}
